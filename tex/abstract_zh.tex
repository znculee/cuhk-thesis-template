% !TEX root = ../main.tex

\newcommand{\forceindent}{\leavevmode{\parindent=1em\indent}}

\pagestyle{plain}

\chapter*{摘要}
\addcontentsline{toc}{chapter}{摘要}
\vspace*{-10mm}

\forceindent
【中大簡介】
香港中文大學(中大)成立於1963年,為研究型綜合大學,以「結合傳統與現代,融會中國與西方」為使命,蹈厲奮發,志在千里。中大師生來自世界各地。我們也有廣大的本地和海外校友組織,聯繫身在世界各地的中大畢業生。

【優質教學】
中大是香港乃至亞洲首屈一指的大學,本校的宗旨是培育既具專精知識又有處世智慧的人才,本校特色包括靈活學分制、書院制、中英兼重和多元文化;並特設通識教育 ,以拓寬學生視野,及培養綜合思考能力,使學生在瞬息萬變的現代社會中,能內省外顧,成為出色的領袖人才,貢獻社會。中大的八個學院 提供林林總總的本科 和研究院課程.

【研究稱譽】
香港中文大學研究項目包羅萬象,遍及各個學科。校方又予教員自由為業界提供顧問服務或與之協作之便。在嚴格的自我要求下,大學的研究一直保持上乘水準,享譽日隆。香港大學教育資助委員會(教資會)選定了二十四個卓越學科領域,集中資源資助本地大學進行研究,其中九個由中大學者負責。現時中文大學有五間由中國科學技術部批准成立的國家重點實驗室,具備國際一流水平的研究能力,完成國家交付的科研重任。在發表研究成果方面,中大的成績粲然可觀。無論在專門領域的學報,還是一般人耳熟能詳的期刊,如《自然》、《科學》、《刺針》,都可見中大學人的文章。

【獨有的書院制】
書院制是中大特色,在本港大學中獨一無二。現有的成員書院計有崇基學院、新亞書院、聯合書院和逸夫書院,和新增的晨興書院、善衡書院、敬文書院、伍宜孫書院及和聲書院。它們與大學相輔相成,提供以學生為本的全人教育和關顧輔導,加強師生間的交流和互動,凝聚學生對書院和母校的歸屬感。

【校園環境】
中大校園面積一百三十七點三公頃,俯瞰吐露港,是全港最寬廣、最綠意盎然的校園。為滿足學習與生活所需,校內有齊備的設施 ,包括一流的圖書館,另有文物館、音樂廳、游泳池、運動場、網球場、壁球場、水上活動中心和健身室等。

\vspace*{15mm}
\cleardoublepage

\let\forceindent\undefined
